\documentclass[12pt, a4paper]{report}
\usepackage[spanish]{babel}
\usepackage[utf8]{inputenc}
\usepackage{graphicx} 
\usepackage{amsmath}
\usepackage{algorithm}
\usepackage[noend]{algpseudocode}

\makeatletter
\def\BState{\State\hskip-\ALG@thistlm}
\makeatother

\graphicspath{ {images/} }

\title{%
\textbf{Simulación de Sistemas}\\
Trabajo Práctico 2: Autómata Celular Off-Lattice \\
\large \emph{Bandadas de Agentes Autopropulsados}
}

\author{Piñeiro, Eugenia Sol - 59449\\
Baiges, Matías Sebastián - 59076\\
Comerci Wolcanyik, Nicolás - 59520
}
\date{3 de septiembre de 2021}

\begin{document}

\maketitle

\tableofcontents
\newpage

\section{Introducción}
El objetivo del siguiente trabajo práctico es implementar un Autómata Celular \emph{Off-Latice}, 
a fin de analizar el comportamiento de partículas puntuales en base a distintos parámetros.\\

El autómata \emph{Off-Latice} se basa en un modelo matemático inspirado en el comportamiento colectivo de algunos Sistemas Naturales. Estos sujetos biológicos tienden a moverse con sus vecinos. Por ejemplo, bandadas, cardúmenes, manadas de cuadrúpedos o una colonia de bacterias.\\

En cuanto al sistema, las partículas se desplazan con velocidad absoluta constante. Las mismas, en cada paso de tiempo, asumen el promedio de las direcciones de las partículas vecinas que se encuentran dentro de un radio \emph{r} con un ruido añadido de forma aleatoria.\\

Este modelo exhibe un comportamiento cooperativo entre las partículas, las cuales a medida que va avanzando el tiempo comienzan a moverse con una simetría de forma espontánea.\\ 

La evolución temporal de las partículas viene dada por las siguientes ecuaciones: 
\begin{equation}
\label{eq:velocity}
x_i(t+1) = x_i(t) + v_i(t) \Delta t 
\end{equation}

donde \emph{x} es la posición de la partícula, \emph{v} la velocidad, y $\Delta$ \emph{t} el paso de tiempo  

\begin{equation}
\label{eq:angle}
\theta _i(t+1) = \left\langle \theta _i(t)\right\rangle _r + \Delta \theta 
\end{equation}
donde $\Delta \theta  \epsilon  [-\frac{\eta}{2}; \frac{\eta}{2}]$ es el ruido añadido, siendo $\eta$ la amplitud del ruido, y \emph{$\left\langle \theta _i(t)\right\rangle _r$} es el promedio de los ángulos de todas las partículas en un radio \emph{r}

\begin{equation}
\label{eq:avgangle}
\left\langle \theta _i(t)\right\rangle _r =  \frac{\left\langle sin(\theta _i(t))\right\rangle _r}{\left\langle cos(\theta _i(t))\right\rangle _r} 
\end{equation}

\section{Implementación}
La implementación del algoritmo \emph{Off-Lattice} se desarrolló en el lenguaje de programación \emph{Java}\\

El flujo principal del programa (ver Figura \ref{fig:diagrama_flujo}) consiste en diversos pasos. En primer lugar, se obtienen los parámetros de configuración para iniciar la Simulación.\\

Luego, se generan partículas de forma aleatoria y se van actualizando tanto sus posiciones como sus ángulos con el algoritmo del autómata. \\

Por último, se guardan los resultados en distintos archivos que serán utilizados como \emph{input} para generar la Animación en \emph{Ovito} y realizar el postprocesamiento en \emph{Python} para hacer un análisis de resultados.\\

\begin{figure}[h]
\includegraphics[scale=0.35]{Diagrama de flujo.png}
\centering 
\caption{Diagrama de Flujo}
\label{fig:diagrama_flujo}
\end{figure}

En cuanto a la implementación de la Simulación, se cuenta con distintas clases en Java (ver Figura \ref{fig:uml}). La clase que implementa el flujo principal (main) es \emph{OffLatticeSimulation}. Las partículas se modelaron con la clase \emph{VelocityPaticle}, que extiende de una clase \emph{Particle} y cuenta con su posición, ángulo y rapidez. Las mismas se generan a partir de la clase \emph{VelocityParticleGenerator}.\\

Para calcular la polarización de las mismas se cuenta con la clase \emph{Polarization}. El algoritmo del autómata se implementó en la clase \emph{OffLattice} que hace uso de las mencionadas anteriormente.\\  

A su vez, se cuenta con las clases para obtener los parámetros de configuración y obtención de resultados \emph{Config, XYZWriter, JsonWriter} y \emph{Postprocessing}, sobre las cuales no se entrará en detalle ya que corresponden al postprocesamiento.

\pagebreak

\begin{figure}[h]
\includegraphics[scale=0.3]{UML.png}
\centering 
\caption{Diagrama UML}
\label{fig:uml}
\end{figure}

En el pseudocódigo correspondiente al flujo principal del programa (ver Algoritmo 1) se puede notar que se sigue el flujo mencionado previamente, en el cual se llama a la función \emph{readValues()} para obtener la configuración del sistema, luego se generan partículas de forma aleatoria invocando la función \emph{generateRandom()}, se llama al autómata Off-Lattice y por último se guardan los resultados en los archivos correspondientes.\\

En cuanto a la implementación del autómata (ver Algoritmo 2) cabe destacar que por cada pasa de tiempo (frame), en la línea 3 del pseudocódigo, se buscan los vecinos de las partículas a través del \emph{Cell Index Method}. Este método define una grilla de dimensión MxM y se utilizan condiciones de contorno periódicas \cite{cellIndex}
Luego, se actualizan el ángulo y la posición de cada partícula utilizando las ecuaciones de Evolución Temporal (1) (2) y (3). 

\pagebreak
\begin{algorithm}
    \caption{Off-Lattice Simulation Main}\label{main}
    \begin{algorithmic}[1]
    \Procedure{main}{}
    \State $\textit{config} \gets \text{mapper.readValues()}$
    \State $particles \gets \textit{VelocityParticlesGenerator.generateRandom(config)}$
    \State $frames \gets \textit{OffLattice.simulate(config)}$
    \State $XYZ_Writer(FILENAME).addAllFrames(frames)$
    \EndProcedure
    \end{algorithmic}
\end{algorithm}
\begin{center}
Algoritmo 1. Flujo Principal del Programa
\end{center}

\begin{algorithm}
    \caption{Off-Lattice Algorithm}\label{offLattice}
    \begin{algorithmic}[1]
    \Procedure{simulate}{}
    \BState \emph{\textbf{for} frame in frames}:
    \State $\textit{neighbours} \gets \text{CellIndexMethod.search()}$
    \BState \emph{\textbf{for} particle in neighbours + itself}:
    \State $\textit{newAngle} \gets \text{atan2(sinAvg, cosAvg) + noise}$
    \State $particle.setAngle(newAngle)$
    \State $particle.setCoords((x + speed * cos(newAngle)) \% L,(y + speed * sin(newAngle)) \% L)$
    \EndProcedure
    \end{algorithmic}
\end{algorithm}
\begin{center}
Algoritmo 2. Algoritmo del Autómata Off-Lattice
\end{center}

\section{Simulaciones}

El esquema del sistema (ver Figura \ref{fig:esquema}) consta de una grilla cuyo lado tiene longitud \emph{L}, en la cual interactúan partículas con velocidad constante \emph{v} e interactúan dentro un radio \emph{r}.

\begin{figure}[h]
\includegraphics[scale=0.42]{esquema.png}
\centering 
\caption{Esquema del Sistema}
\label{fig:esquema}
\end{figure}

Para realizar las simulaciones se utilizaron los siguientes parámentros: 
\begin{itemize}
    \item radio de interacción entre partículas \emph{r = 1}
    \item velocidad absoluta de las partículas \emph{v = 0.03}
    \item amplitud del ruido \emph{$\eta$ $\epsilon$ [0; 2$\pi$]} 
    \item lado de la grila L (Variable) 
    \item cantidad de partículas N (Variable)
    \item densidad $\rho = \frac{N}{L^2}$
    \item cantidad de pasos \emph{frames} = 2000  
\end{itemize}

En cada simulación se observó la polarización del sistema, la cual se define con la siguiente ecuación: 
\begin{equation}
    \label{eq:polarization}
     v_a = \frac{1}{Nv} \left\lvert \sum_{i = 1}^{N} v_i \right\rvert
\end{equation}\\

Si la polarización tiende a cero, las partículas se encuentran en total desorden, mientras que si la polarización tiende a uno se dice que las partículas están polarizadas.

Para correr las simulaciones se utilizarán las siguientes configuraciones: 
\begin{enumerate}
    \item Para analizar el ruido se utilizó N=300, $\rho$=0.6 y $\eta$ variable
    \item Para analizar la densidad se utilizó N=300, $\eta$=0.8 y $\rho$ varibale 
    \item Densidad y ruido pequeños:  N = 300, L = 25, ($\rho$ = 0.48), $\eta$ = 0.1
    \item Densidad y ruido grandes:  N = 300, L = 7, ($\rho$ = 6.12), $\eta$ = 2.0  
    \item Densidad grande y ruido pequeño: N = 300,	L = 5, ($\rho$ = 12), $\eta$ = 0.1
\end{enumerate}


\section{Resultados}

A partir de las configuraciones mencionadas previamente, se obtuvieron los siguientes resultados tras analizar el efecto del ruido y la densidad.

\subsection{Análisis del Ruido}

Luego de realizar simulaciones bajo las configuraciones propuestas (ver Figuras \ref{fig:noise_N300_d06_n05}, \ref{fig:noise_N300_d06_n1_5} y \ref{fig:noise_N300_d06_n5}), se observó que a medida que el ruido aumentaba, las particulas tendían a dispersarse, dirigiendose hacia distintas direcciones. Es decir, manteniendo la cantidad de partículas y el ruido constante, \emph{al aumentar el ruido, la polarización disminuye}.\\

En caso de poco ruido (ver Figura \ref{fig:noise_N300_d06_n05}), se pudo observar como las partículas al cabo de un par de pasos (\emph{frames}) se movían en la misma dirección (\emph{partículas altamente polarizadas}).\\

\begin{figure}[h]
\includegraphics[scale=0.4]{noise_N300_d06_n05.png}
\centering 
\caption{Capturas de los frames inicial, intermedio y final de una Animación con N=300; densidad=0.6 y ruido=0.5}
\label{fig:noise_N300_d06_n05}
\end{figure}

\begin{figure}[h]
\includegraphics[scale=0.4]{noise_N300_d06_n1_5.png}
\centering 
\caption{Capturas de los frames inicial, intermedio y final de una Animación con N=300; densidad=0.6 y ruido=1.5}
\label{fig:noise_N300_d06_n1_5}
\end{figure}

\pagebreak
\begin{figure}[h]
\includegraphics[scale=0.4]{noise_N300_d06_n5.png}
\centering 
\caption{Capturas de los frames inicial, intermedio y final de una Animación con N=300; densidad=0.6 y ruido=5}
\label{fig:noise_N300_d06_n5}
\end{figure}

También se observó este mismo fenómeno al comparar la evolución de la polarización en función del tiempo para distintos valores de ruidos (ver Figura \ref{fig:noise_pola_vs_time_multiple_n}). Además, se observó esta situación para distintos valores de ruido (ver Figura \ref{fig:pola_vs_noise}), y para distintas cantidades de partículas (ver Figura \ref{fig:pola_vs_noise_multiple_n}), donde vimos que mientras menos partículas, mayor era la polarización manteniendo el ruido constante.

\begin{figure}[h]
\includegraphics[scale=0.45]{noise_pola_vs_time_multiple_n.png}
\centering 
\caption{Polarización en función del tiempo para múltiples valores del ruido, $\rho$=0.6, N=300 }
\label{fig:noise_pola_vs_time_multiple_n}
\end{figure}

\pagebreak
\begin{figure}[h]
\includegraphics[scale=0.5]{pola_vs_noise.png}
\centering 
\caption{Polarización en función del ruido para $\rho$=0.6, N=300 y considerando el frame 1200 como comienzo del régimen estacionario.}
\label{fig:pola_vs_noise}
\end{figure}

\begin{figure}[h]
\includegraphics[scale=0.5]{pola_vs_noise_multiple_n.png}
\centering 
\caption{Polarización en función del ruido para múltiples valores de N con $\rho$=0.6 y considerando el frame 1200 como comienzo del régimen estacionario.}
\label{fig:pola_vs_noise_multiple_n}
\end{figure}

\subsection{Análisis de la Densidad}

Luego de realizar simulaciones bajo las configuraciones propuestas (ver Figuras \ref{fig:density_N300_n08_d025}, \ref{fig:density_N300_n08_d3} y \ref{fig:density_N300_n08_d6}), se observó que a medida que se aumentaba la densidad de partículas, las partículas tienden a moverse de manera más ordenada. Es decir, manteniendo la cantidad de partículas y el ruido constantes, \emph{al aumentar la densidad de partículas, la polarización aumenta}.

\pagebreak
\begin{figure}[h]
\includegraphics[scale=0.54]{density_N300_n08_d025.png}
\centering 
\caption{Capturas de los frames inicial, intermedio y final de una Animación con N=300; densidad=0.25 y ruido=0.8}
\label{fig:density_N300_n08_d025}
\end{figure}

\begin{figure}[h]
\includegraphics[scale=0.4]{density_N300_n08_d3.png}
\centering 
\caption{Capturas de los frames inicial, intermedio y final de una Animación con N=300; densidad=3 y ruido=0.8}
\label{fig:density_N300_n08_d3}
\end{figure}

\pagebreak
\begin{figure}[h]
\includegraphics[scale=0.4]{density_N300_n08_d6.png}
\centering 
\caption{Capturas de los frames inicial, intermedio y final de una Animación con N=300; densidad=6 y ruido=0.8}
\label{fig:density_N300_n08_d6}
\end{figure}

% TODO: Agregar descripción sobre los gráficos de abajo

\pagebreak
\begin{figure}[h]
\includegraphics[scale=0.5]{density_pola_vs_time.png}
\centering 
\caption{Polarización en función del tiempo para múltiples valores de densidad con N=300; $\eta$=0.8}
\label{fig:density_pola_vs_time}
\end{figure}

\pagebreak
\begin{figure}[h]
\includegraphics[scale=0.5]{pola_vs_density.png}
\centering 
\caption{Polarización en función de la densidad con N=300; $\eta$=0.8 usando el frame 1200 como comienzo del régimen estacionario.}
\label{fig:pola_vs_density}
\end{figure}

\subsection{Comportamientos Destacados}

Se utilizaron distintas configuraciones propuestas en el paper \cite{vicsek1995novel} con el objetivo de evaluar distintos comportamientos destacados del sistema.\\

En primer lugar, se utilizaron valores de densidad y ruido pequeños y se pudo notar que las partículas tienden a formar grupos moviendose de forma coherente en direcciones aleatorias (ver Figura \ref{fig:d_n_small})  
\pagebreak
\begin{figure}[h]
\includegraphics[scale=0.4]{d_n_small.png}
\centering 
\caption{Capturas de los frames inicial, intermedio y final de una Animación con  N = 300; L = 25;	($\rho$ = 0.48); $\eta$ = 0.1}
\label{fig:d_n_small}
\end{figure}

Luego, para valores de densidad y ruido grandes se notó que las partículas se desplazan de forma aleatoria pero con cierta correlación.  
\pagebreak
\begin{figure}[h]
\includegraphics[scale=0.4]{d_n_big.png}
\centering 
\caption{Capturas de los frames inicial, intermedio y final de una Animación con N = 300; L = 7; ($\rho$ = 6.12); $\eta$ = 2.0}
\label{fig:d_n_big}
\end{figure}

Por último, con una densidad grande y un ruido pequeño se notó que el movimiento se vuelve ordenado para todas las partículas.
\pagebreak
\begin{figure}[h]
\includegraphics[scale=0.4]{d_big_n_small.png}
\centering 
\caption{Capturas de los frames inicial, intermedio y final de una Animación con N = 300; L = 5; ($\rho$ = 12); $\eta$ = 0.1}
\label{fig:d_big_n_small}
\end{figure}

\section{Conclusiones}
Tomando como base los resultados obtenidos a partir de las simulaciones, se llegó a la 
conclusión que, al aumentar el ruido disminuye la polarización y al aumentar la densidad de
partículas aumenta la polarización.\\

A su vez, se puede observar que el sistema presenta distintos comportamientos según los parámetros elegidos. 

\bibliographystyle{plain}
\bibliography{references}

\end{document}